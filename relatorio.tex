\documentclass[a4paper,10pt]{coursepaper}
\usepackage[utf8]{inputenc}
\usepackage[brazil]{babel}

%opening
\title{Comunicação entre dois computadores através de RAW Socket}
\author{Antonio Carlos Salzvedel Furtado Junior e Tiago Rodrigo Kepe}
\studentnumber{GRR20080946,GRR20084630}
\college{Universidade Federal do Paraná}
\coursename{Bacharelado em Ciência da Computação}
\coursenumber{CI058}
\coursesection{Redes I}
\instructor{Luiz Carlos Pessoa Albini}


\begin{document}

\maketitle

\section{Funcionamento}

O funcionamento de nossa rede funciona no esquema cliente-servidor. A máquina que estiver executando o servidor ficará escutando a rede,  esperando por mensagens do cliente. Quano o cliente for executado, uma espécie de Shell estará disponível para o usuário. Todos os comandos definidos na especificação estarão disponíveis, são eles:
\begin{itemize}
 \item ls [opções]
 \item cd
 \item lls [opções]
 \item lcd
 \item put
 \item get
\end{itemize}


\section{Implementação}

O envio de mensagens não é feito com janelas deslizantes, como definido na especificação. A nossa implementação baseia-se no ``Para e espera''. 

Tratamos o caso em que uma mensagem é esperada, e por algum motivo ela for perdida, seja por retirada do cabo ou outro motivo. A máquina que quiser receber a mensagem, seja cliente ou servidor, deverá esperar 16 vezes por um determinado tempo (TIMEOUT). Para cada uma das tentativas falhas de recebimento, ela deverá enviar um Nack. No final da 16ª vez ela deve desistir de receber a mensagem.
Quem estiver enviando, deve tentar enviar até receber uma um Ack. A tentativa de recebimento de resposta segue o mesmo modelo citado, exceto que nenhum Nack é enviado, já que você entraria em um ciclo de recebimento de respostas de respostas.


O Timeout é feito por uma função. Ela usa a função do sistema chamada select. Esta função monitora o descritor do nosso socket por um determinado período. Ela retorna positivo se alguma mensagem foi recebida durante a constante TIMEOUT definida, mesmo que seja lixo, e falso se durante este período nada foi recebido.

O campo tipo de mensagem define todos os tipos definidos na especificação. Além disso foram definidos tipos específicos para mensagens de erro, por motivo de simplicidade de implementação, são eles:
\begin{itemize}
 \item TYPE\_E1 : diretório inexistente;
 \item TYPE\_E2 : falta de permissão;
 \item TYPE\_E3 : espaço insuficiente;
 \item TYPE\_E4 : arquivo inexistente.
\end{itemize}

A paridade feita é vertical

\section{Classes}

Realizamos nosso trabalho em C++, então dividimos funções específicas para cada uma de nossas classes, são elas:
\begin{itemize}
 \item Message:
 Esta classe é responsável por definir todos os detalhes de nossas mensagens. Ela possui funções para montagem de mensagem, gerador e conferidor de paridades e deve retornar campos específicos da mensagem.
 \item Socket:
 Ela é reponsável por abrir o socket de acordo com regras do RAW SOCKET. Ele possui funções simples para envio e recebimento de mensagens simples.
 \item Control:
 Esta classe cuida de detalhes de maior abstração da rede. Ela usa a classe Socket para enviar e receber múltiplas mensagens, ela também cuida de detalhes como o TIMEOUT da rede e da seqüêncialização de mensagens. Os detalhes do para e espera são definidos aqui também.
 \item Cliente:
 Ela define o comportamento do cliente. Basicamente, é criado um shell e funções relacionadas aos comandos de especificação. Ela usa a classe Control para cuidar do resto.
 \item Servidor:
 Assim como o cliente, ela define o comportamento do servidor. Também usa a classe Control. A sua diferença é não possuir um shell, e sim uma função para escutar na rede.
\end{itemize}

\end{document}
